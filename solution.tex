\documentclass{ctexbeamer}

\hypersetup{
  pdfcreator=TeX,
  pdfpagemode=FullScreen,
}

\mode<presentation>{
  \usetheme{Madrid}
  \setbeamertemplate{navigation symbols}{}
  \setbeamertemplate{footline}[frame number]
}

% -------------- 文档信息 --------------
\title{数据库大作业展示}
\subtitle{商店系统-图书管理系统}
\author{计试2201 梁佳伟 \and 计试2201 郑诗棪}
\institute[西安交大]{西安交通大学}
\date{2025年5月}
\subject{数据库课程展示}

\begin{document}

% 首页
\begin{frame}
  \titlepage
\end{frame}

% 目录
\begin{frame}{目录}
  \begin{itemize}
    \item \extbf{设计思想}
    \begin{itemize}
      \item 整体设计思想
      \item 百货商店概览
      \item 图书管理系统
    \end{itemize}
    \item \textbf{系统展示}
    \item \textbf{实现方案}
    \item \textbf{总结}
  \end{itemize}
\end{frame}

% 设计思想
\section{设计思想}

\subsection{整体设计思想}
\begin{frame}{整体设计思想}
  \begin{itemize}
    \item 构建百货商店管理后台,涵盖商品、部门、员工、制造商等功能
    \item 在此框架下,深入开发“图书管理系统”模块,面向用户呈现
    \item 图书系统支持:
      \begin{itemize}
        \item 图书信息展示与分页浏览
        \item 多条件搜索(书名+作者)
        \item 可视化统计分类数据
        \item 购买需求清单(添加/删除)
      \end{itemize}
  \end{itemize}
\end{frame}

\subsection{百货商店概览}
\begin{frame}{百货商店概览}
  \begin{columns}
    \column{0.6\textwidth}
      \begin{itemize}
        \item 商品管理:增删改查、分类管理
        \item 部门管理:录入部门、删除编辑部门、部门商品联系
        \item 制造商管理:制造商商品联系、制造商信息记录
        \item 员工管理:员工信息录入、员工和部门关系记录
      \end{itemize}
  \end{columns}
  \begin{columns}
      \column{0.45\textwidth}
      \begin{figure}
        \centering
        % ER 图示例
        \includegraphics[width=0.9\textwidth]{fig/ER.png}
        \caption{百货商店 ER 图}
      \end{figure}
  \end{columns}
\end{frame}

\subsection{图书管理系统}
\begin{frame}{图书管理系统}
  \begin{itemize}
    \item 爬虫从豆瓣获取约3万条图书数据
    \item 存储字段:书名、作者、分类、价格、页数、评分、简介、详情链接
    \item 用户可:搜索、浏览、查看详情、添加“购买需求”、查看\&删除需求
    \item 支持分类统计与可视化展示
  \end{itemize}
  \vspace{0.3cm}
  \begin{figure}
    \centering
    \begin{minipage}{0.3\textwidth}
      \centering
      \includegraphics[height=2cm]{fig/bookJSON.png}
      \caption{图书 JSON 示例}
    \end{minipage}
    \hfill
    \begin{minipage}{0.45\textwidth}
      \centering
      \includegraphics[height=3cm]{fig/bookER.png}
      \caption{图书 ER 图}
    \end{minipage}
  \end{figure}
\end{frame}

% 前端展示
\section{前端展示}

\subsection{百货商店入口}
\begin{frame}{百货商店入口}
  \begin{figure}
    \centering
    \includegraphics[width=0.7\textwidth]{fig/store1.png}
    \caption{百货商店后台首页}
  \end{figure}
  
\end{frame}

\subsection{百货商店部分功能}
\begin{frame}{百货商店部分功能}
  \centering
  \includegraphics[width=0.45\textwidth]{fig/store4.png}
  \hspace{0.5cm}
  \includegraphics[width=0.45\textwidth]{fig/store3.png}
  \includegraphics[width=0.45\textwidth]{fig/store5.png}
  \hspace{0.5cm}
  \includegraphics[width=0.45\textwidth]{fig/store6.png}
  \vspace{0.3cm}
\end{frame}

% 图书管理系统前端
\subsection{图书管理系统部分功能}
\begin{frame}{图书管理系统部分功能}
  \centering
  \begin{minipage}{0.45\textwidth}
    \includegraphics[width=\textwidth]{fig/book1.png}\\ 图书列表与分页
  \end{minipage}
  \begin{minipage}{0.45\textwidth}
    \includegraphics[width=\textwidth]{fig/book2.png}\\ 搜索图书
  \end{minipage}
  \vspace{0.3cm}
  \begin{minipage}{0.45\textwidth}
    \includegraphics[width=\textwidth]{fig/book3.png}\\ 详情跳转
  \end{minipage}
  \begin{minipage}{0.45\textwidth}
    \includegraphics[width=\textwidth]{fig/book4.png}\\ 添加购买需求
  \end{minipage}
\end{frame}

\begin{frame}{图书管理系统更多功能}
  \centering
  \begin{minipage}{0.45\textwidth}
    \includegraphics[width=\textwidth]{fig/book5.png}\\ 购买清单查看
  \end{minipage}
  \begin{minipage}{0.45\textwidth}
    \includegraphics[width=\textwidth]{fig/book6.png}\\ 分类统计可视化
  \end{minipage}
  \vspace{0.3cm}
  \begin{itemize}
    \item 用户点击图书可跳转至豆瓣页面获取详细信息
    \item 使用 ECharts 生成图书分类柱状图,直观呈现图书分布
  \end{itemize}
\end{frame}

% 后端思路
\section{后端思路}

\subsection{数据库设计}
\begin{frame}{数据库设计}
  \begin{itemize}
    \item 核心表:\texttt{books}, \texttt{authors}, \texttt{big_categories}, \texttt{purchase_requests}
    \item 主外键关系:\\
      \texttt{books.author\_id} \ensuremath{\to} \texttt{authors.id}\\
      \texttt{books.big\_class\_id} \ensuremath{\to} \texttt{big_categories.id}\\
      \texttt{purchase\_requests.book\_id} \ensuremath{\to} \texttt{books.id}
    \item 购书需求记录附带 \texttt{username} 与时间戳
  \end{itemize}
\end{frame}

\subsection{Python–Flask 实现}
\begin{frame}{Python–Flask 实现}
  \begin{itemize}
    \item \textbf{Flask 简介}:Flask 是 Python 编写的轻量级 Web 框架,基于 Werkzeug 工具箱和 Jinja2 模板引擎,适合快速开发 Web 应用。
    \item \textbf{工作原理}:
      \begin{itemize}
        \item 浏览器向 Flask 服务器发送 HTTP 请求
        \item Flask 根据 URL 路由分发请求,调用相应函数处理逻辑
        \item 将数据传递给 HTML 模板渲染后返回给浏览器
      \end{itemize}
    \item 使用 \texttt{Flask} 与 \texttt{sqlite3} 实现后端服务
    \item 统一数据库连接 \& 事务管理:\texttt{get\_db()}, \texttt{teardown\_appcontext}
    \item 路由设计:
      \begin{itemize}
        \item \texttt{/books}:分页浏览
        \item \texttt{/search}:模糊搜索
        \item \texttt{/add\_to\_purchase/<book\_id>}:添加购买需求
        \item \texttt{/purchase\_list}:查看 \& 删除需求
      \end{itemize}
    \item 登录无需密码,仅录入用户名并创建需求表
  \end{itemize}
\end{frame}

\subsection{Springboot+Vue实现}
\begin{frame}{Springboot+Vue实现}
\begin{itemize}
    \item \textbf{前端技术栈:Vue 3}
    \begin{itemize}
        \item 使用最新的 Vue 3 框架
        \item 采用 Composition API 风格
        \item  使用 Vite 作为构建工具
        \item Axios 用于处理 HTTP 请求,与后端 API 进行通信,处理数据交互
        \item Element Plus 使用 Element Plus 作为 UI 组件库
    \end{itemize}
    \item \textbf{后端技术栈:Spring Boot}
    \begin{itemize}
        \item 基于 Spring Boot 框架
        \item 提供 RESTful API
        \item  自动配置和依赖管理
    \end{itemize}
\end{itemize}
\end{frame}

\section{总结}

\begin{frame}{总结}
  \begin{itemize}
    \item 项目集成百货商店与图书管理两大模块,涵盖增删改查与可视化统计
    \item 图书系统:3万条数据采集,支持分页浏览、搜索、详情跳转与购买需求管理
    \item 前端:Vue3 + ECharts,交互友好;后端:Flask/SQLite 与 Spring Boot RESTful API 并行
    \item 数据库设计合理,核心表关联清晰,事务管理与路由设计完善
    \item 系统功能齐全、易用性好,可进一步优化用户认证与大数据性能
  \end{itemize}
\end{frame}

\begin{frame}
  \begin{center}
    {\Huge \textbf{谢谢大家}}
  \end{center}
\end{frame}

\end{document}

